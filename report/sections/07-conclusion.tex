\section{Conclusion}\label{sec:conclusion}

In this work, we have presented a temporal \ac{HTN} domain definition for \textit{Overcooked}.
This domain represents a challenging environment for planning, as it requires tasks that are partially ordered.
We find that \ac{FAPE} cannot plan two dish orders at once, similar to the observation for classical planning in the same domain \citep{yuxinliuPlanningOvercookedGame2020}.

Next, we have introduced an actor model that is an extension of \ac{FAPE}s actor model \citep{bit-monnotTemporalHierarchicalModels2016a}.
This extension enables the explicit handling of a stream of tasks.
We introduced two new strategies to anticipate the receiving of new tasks.

First, we defined a robustness heuristic as a plan selection strategy.
This heuristic is supposed to incentivize plans that have a shorter makespan and make the addition and interleaving of new tasks more successful.
We find that it has a small impact on the load distribution between cooks in the \textit{Overcooked} domain.
It is also able to find solutions when the makespan heuristic is not.
We do however observe a high variance in the planning time for complex problems.

Second, we define the concept of preparation insertion in \ac{HTN} planning.
Due to the definition of subtasks in \ac{HTN} planning, we can find subtasks that can be prepared.
We find that the insertion of preparations likely results in more successful executions compared to not using them.
We also find that the success of this preparation insertion relies very much on the preparation selection strategy

We identify several points for further research in this context:

\begin{itemize}
\item Our domain definition for \textit{Overcooked} can be extended to include more of the concepts in the game such as collisions, fires breaking out or washing of used dishes.
Eventually, we believe that this domain can serve as a relevant benchmark for the combination of planning and acting.

% Acting implementation not working
\item Due to missing features in the \ac{FAPE} planner, the implementation of the actor framework is not fully working and was therefore not evaluated.
This has to be further investigated, for instance by using a different planner or adding the missing features, in particular timeline splitting and action removal or the definition of existing actions in the planning problem, to \ac{FAPE}.

% Plan Repair and Replanning
\item While replanning in \ac{HTN} planning has already been evaluated, there do not exist any studies yet that focus on plan repair in \ac{HTN} planning.
These two strategies should be further evaluated to find if the statements from classical planning apply to \ac{HTN} planning.
\end{itemize}

% Acting Competition
Lastly, we have found that resources on acting are most of the time either theoretical or domain-dependent.
We think that this may be connected to the fact that the \ac{IPC} only focuses on different planning variants, while most of the other competitions in this context are domain specific competitions and challenges that require a complete system with domain-dependent implementations.
We suggest that it may be time for a middle ground with acting benchmarks and competitions that incentivize domain-independent systems.
