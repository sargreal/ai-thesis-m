\section{Experimental Evaluation}\label{sec:evaluation}

An Acting system, the Robustness Heuristics and the Task Perperations were implemented directly inside the system Fape \citep{bit-monnotFAPEConstraintbasedPlanner2020}.
The Statement around Fape supporting acting made in \cite{bit-monnotTemporalHierarchicalModels2017} were not present in the current system as they have apperently only been discussed theoretically or parts were deleted because of the domain dependent implementation.
Due to technical difficulties, planning of tasks by interleaving with already planned tasks and replanning could not be implemented in time.
Evaluation regarding these parts is done manually.

The evaluation was done in a LXC container with 12 Intel(R) Xeon(R) Gold 6334 CPU 3.60GHz Cores, 32 GB RAM.
The planning itself is however a single threaded process, so only one CPU Core can be used.

It has to be mentioned that Fape is not the most performant HTN planner, but besides CHIMP the only one supporting all required features.

Quantitative Metrics:
Plan time
Search Nodes
Makespan/Duration
Average Branching Factor

Qualitative Metrics:
Success

Possible Comparisons:
HTN PO vs HTN TO vs Classical \citep{yuxinliuPlanningOvercookedGame2020}
HTN vs HTN + Robustness
HTN Oracle vs HTN Acting vs HTN Acting + Robustness vs HTN + Preparations


\subsection{HTN vs Classical Planning in Overcooked}

\bild{fig:eval-mapB}{pddl_mapB}{Map}{10}

The standard Fape Planner is compared to the Temporal Fast Downward Planner used in \cite{yuxinliuPlanningOvercookedGame2020} in the Domain Overcooked.
The problem considered is the map B from \cite{yuxinliuPlanningOvercookedGame2020} shown in figure \ref{fig:eval-mapB}.
They call it a complex map, as it incentivices the cooks to use the counters for faster transportation.
The tasks in the domain are to prepare one or two burgers.
The task for creating a burger is modeled as both a Partial Order and a Total Order network in HTN.