\section{Discussion}\label{sec:discussion}

In this section, we will discuss the results of the evaluation and thus the validity of our approach.
Additionally, we go into some of the issues that we encountered with the system FAPE.
Lastly, we will discuss the replan and repair approach, as that could not be evaluated. 


Planning using the fully hierarchical domain proves to be more efficient than a partial hierarchy.


The robustness heuristic we introduced does not perform as good as we had proposed.
While it succeeds at parallelizing tasks, it has arbitrary waiting times included.
This problem comes probably from the fact that many options for variable bindings are kept in place for too long.
During planning, the timelines are created incrementally.
When threats occur between timelines, the timelines are separated.
When the variable bindings are then resolved to be ground, these separations are kept in place and cannot be changed anymore.
Through the robustness heuristic, some of the variable bindings are resolved earlier, but not enough.

Additionally, the heuristic results in high variance for the planning time and also the makespan.
This variance is most likely related to the low value of 10 we chose for $n$ and $m$.
While a more thorough evaluation could reveal the best values for these parameters, we already see an decrease by over  of the number of nodes generated in the same 
This high variance makes the approach not viable in environments 


\cite{bit-monnotTemporalHierarchicalModels2017} argues that qualitative time points in this context have the only benefit of creating instantaneous action effects, which are not necessary in most environments.

FAPE claims several times to do Acting, but it is not implemented.
Missing features included: 
HTN Repair/Replanning (only classical planning was considered before it was removed from the code). 
Removal of actions and tasks from a plan.
The use of STNUs for contingent time points - technically implemented, but not documented how to use them.
The insertion of timelines in between an existing timeline is not supported.


FAPE included only some parts of the theoretical actor model as a domain-specific implementation that was removed in the stable release.
It was only supported to add new goals before starting dispatching.
Additionally replanning and plan repair were only considered in non-hierarchical domains.

The modification of an existing and already resolved plan poses a challenge in several aspects.
The timelines for all instances are already resolved.
In the FAPE planner it is not possible to insert a state change between timelines, but only after or before them.
This is especially problematic, since then a task with a tight deadline cannot be inserted when an instance needs to be used earlier.


It has to be mentioned that Fape is not the most performant HTN planner, but besides CHIMP the only one supporting all required features.