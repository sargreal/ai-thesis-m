\section{Discussion}\label{sec:discussion}


\cite{bit-monnotTemporalHierarchicalModels2017} argues that qualitative time points in this context have the only benefit of creating instantaneous action effects, which are not necessary in most environments.

FAPE claims several times to do Acting, but it is not implemented.
Missing features included: 
HTN Repair/Replanning (only classical planning was considered before it was removed from the code). 
Removal of actions and tasks from a plan.
The use of STNUs for contingent time points - technically implemented, but not documented how to use them.
The insertion of timelines in between an existing timeline is not supported.


FAPE included only some parts of the theoretical actor model as a domain-specific implementation that was removed in the stable release.
It was only supported to add new goals before starting dispatching.
Additionally replanning and plan repair were only considered in non-hierarchical domains.

The modification of an existing and already resolved plan poses a challenge in several aspects.
The timelines for all instances are already resolved.
In the FAPE planner it is not possible to insert a state change between timelines, but only after or before them.
This is especially problematic, since then a task with a tight deadline cannot be inserted when an instance needs to be used earlier.


It has to be mentioned that Fape is not the most performant HTN planner, but besides CHIMP the only one supporting all required features.