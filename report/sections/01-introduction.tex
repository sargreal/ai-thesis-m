\section{Introduction}\label{sec:introduction}


\subsection{Problem Statement}

Most planning algorithms are doing a strict separation between planning and execution.
While that is fine in most domains, it does not work everywhere.
Using online planning, a planner can dynamically receive domain updates and adjust the plan accordingly.
Most variants of online planning consider the case of plan repair or replanning in the case of failures.
In a dynamic domain it is however very common to receive additional tasks that do sometimes need to be executed in a given time window.
While replanning approaches may help in the integration of these additional tasks, they will in especially strict deadlines fail, due to the previous tasks being executed too lazy.
To achieve better success rates, it is necessary to consider the case of new tasks being inserted in the future already during planning.

The problem considered is therefore a planning domain with temporal constraints, a stream of tasks with deadlines and a hierarchical definition of the domain.
Given this problem, the goal is to create adaptations of the planning algorithm that increase the success rate of the execution of a plan in the case of shorter deadlines in the stream of tasks.