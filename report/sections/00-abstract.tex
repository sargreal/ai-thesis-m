\begin{abstract}
  % Planning in dynamic domains is challenging
  HTN planning in a dynamic domain with a-priori unknown tasks and uncontrollable changes is a challenging problem.
  % Interleaving
  We focus in particular on the problem that occurs when the unknown tasks include deadlines as it might require interleaving a new task with an already existing plan.
  % Overcooked 
  We introduce a hierarchical domain definition of the game ``Overcooked!'' as an example of a dynamic domain with a stream of tasks.
  In ``Overcooked!'' the task is to fulfill dish orders with strict deadlines by managing up to four cooks.
  % Stream of Tasks
  We refine the actor model of the HTN planning system FAPE to support a stream of tasks.
  We aim to solve the problem of handling the stream of tasks by introducing two methods that optimize plans by anticipating the addition of new tasks.
  % robustness heuristic and preparation insertion
  First, we present a new heuristic that evaluates the robustness of a plan in the case of action delays that may arise when interleaving the current plan with a new task.
  Second, we propose a method that inserts preparations for possible future tasks into the current plan by using the hierarchal domain definition.
  % Results
  We find that our robustness heuristic only distributes some of the tasks between actors in the domain but does incentivize plans that are robust enough for the insertion of new tasks.
  Our preparation insertions achieve the interleaving of tasks and will succeed more likely in the case of tasks that have stricter deadlines. 
  Both approaches cannot outperform the optimal solution that is acquired by optimizing the overall makespan.
\end{abstract}